\documentclass{report}
\usepackage[vietnamese]{babel}
\usepackage{titlesec}
\usepackage{chngcntr}
\usepackage{geometry}
\usepackage{setspace}
\usepackage{enumitem}
\usepackage{xcolor}
\geometry{
    a4paper,
    left=3cm,
    right=2.5cm,
    top=2cm,
    bottom=2cm,
}
\setstretch{1.5} % Khoảng cách dòng 1.5 lần
\usepackage{fontspec}
\setmainfont{Times New Roman} % Thay đổi font chữ theo mong muốn
%\usepackage{parskip}
%\usepackage{indentfirst}
%\setlength{\parindent}{1.5em}
\setlist[itemize]{listparindent=\parindent, itemindent=\parindent}
\usepackage{indentfirst}
\setlength{\parindent}{0pt} % Loại bỏ thụt đầu dòng
\newcommand{\gachdau}{\hspace*{1.5em}\ignorespaces} % Tạo gạch đầu dòng
\titleformat{\chapter}[display]
{\normalfont\huge\bfseries\centering}{\chaptertitlename\ \thechapter}{20pt}{\Huge}
\begin{document}

\tableofcontents
\clearpage
\chapter{MỞ ĐẦU}
\section{Đặt vấn đề}
\gachdau
Khái quát về CNH nói chung và CNH ở Việt Nam....\textcolor{red}{(tiếp tục triển khai)}\\
\gachdau
Sự phát triển của AI....\textcolor{red}{(tiếp tục triển khai)}\\
\gachdau
Làm sao để có thúc đẩy phát triển AI trong bối cảnh của cách mạng công nghiệp lần thứ 4  là chủ đề cần được nghiên cứu đầy đủ, toàn diện ở Việt Nam hiện nay.... \textcolor{red}{(tiếp tục triển khai)}\\
\gachdau
\textcolor{red}{Lưu ý: phần này trình bày tối thiểu 300 từ, tối đa 500 từ.}\\

\section{Đối tượng nghiên cứu và phạm vi nghiên cứu}
\subsubsection{* Đối tượng nghiên cứu:}
\gachdau
Cách mạng công nghiệp lần thứ 4 và công nghệ AI
\subsubsection{* Phạm vi nghiên cứu:}
\gachdau
Thực trạng phát triển AI ở Việt Nam trong giai đoạn từ năm 2017 đến năm 2022.
\section{Phương pháp nghiên cứu}
\gachdau
- Đề tài sử dụng phương pháp nghiên cứu lý luận và thực tiễn, kết hợp các phương pháp biện chứng duy vật, lịch sử - logic, so sánh, phân tích - tổng hợp, số liệu - thống kê, v.v.\\
\gachdau
- Đề tài dựa vào các nguồn tài liệu tham khảo chính thức và uy tín, bao gồm các tác phẩm của C.Mác và các nhà Mác - Lênin, các văn bản của Đảng và Nhà nước, các báo cáo, số liệu thống kê, nghiên cứu khoa học, bài báo, sách, v.v.\\
\gachdau
- Phương pháp cơ bản được nhóm sử dụng là: (mỗi nhóm tự xác định phương pháp nghiên cứu chủ yếu được sử dụng trong chủ đề của nhóm mình)
\section{Mục tiêu của đề tài}
\gachdau
* Mục tiêu chung:\\
\gachdau
Phân tích thực trạng phát triển AI của Việt Nam trong quá trình CNH, HĐH và đề xuất những giải pháp nhằm phát triển công nghệ AI đáp ứng yêu cầu của cuộc cách mạng công nghiệp lần thứ 4\\
\gachdau
* Mục tiêu cụ thể:\\
\gachdau
- Làm rõ lý luận về CNH, HĐH của Việt Nam\\
\gachdau
- Phân tích thực trạng công nghệ AI của Việt Nam trong quá trình CNH, HĐH.\\
\gachdau
- Đề xuất các giải pháp nhằm thúc đẩy sự phát triển của công nghệ AI ở Việt Nam trong thời gian tới.
\section{Kết cấu của đề tài}
\gachdau
Ngoài phần kết luận và danh mục tài liệu tham khảo, đề tài được kết cấu thành  3 chương như sau:\\
\gachdau
Chương 1: Mở đầu\\
\gachdau
Chương 2: Lý luận về Công nghiệp hoá, Hiện đại hoá của Việt Nam\\
\gachdau
Chương 3: Phát triển công nghệ AI ở Việt Nam dưới tác động của cách mạng công nghiệp lần thứ tư\\

\textcolor{red}{Chú ý: phần mở đầu viết tối đa 3 trang.}

\chapter{LÝ LUẬN VỀ CÔNG NGHIỆP HOÁ, HIỆN ĐẠI HOÁ Ở VIỆT NAM VÀ CÁCH MẠNG CÔNG NGHHIỆP LẦN THỨ TƯ}

\section{Cách mạng công nghiệp}
\subsection{Khái niệm}
\subsection{Các cuộc cách mạng công nghiệp}
\section{Công nghiệp hoá}
\subsection{Khái niệm}
\subsection{Các mô hình công nghiệp hoá}
\section{Công nghiệp hoá, Hiện đại hoá ở Việt Nam}
\subsection{Khái niệm và đặc điểm CNH, HĐH}
\subsection{Tính tất yếu khách quan của CNH, HĐH}
\subsection{Nội dung của CNH, HĐH ở Việt Nam}
\section{Cách mạng công nghiệp lần thứ tư (4.0)}
\section{Công nghệ AI}
%\clearpage
\chapter{PHÁT TRIỂN CÔNG NGHỆ AI Ở VIỆT NAM DƯỚI TÁC ĐỘNG CỦA CÁCH MẠNG CÔNG NGHIỆP LẦN THỨ TƯ}
\section{Thực trạng phát triển công nghệ AIc của Việt Nam.}
\textcolor{red}{Chú ý: Tìm số liệu thống kê từ các nguồn chính thống như: Tổng cục thống kê, Bộ lao động thương binh và xã hội, Bộ kế hoạch và đầu tư, Bộ Khoa học và Công nghệ, Ngân hàng thế giới (WB)...để thống kê về: quy mô, chất lượng, đầu tư... của đào tạo nguồn nhân lực ở Việt Nam. Trên cơ sở đó lập các bảng biểu, biểu đồ để phân tích, đánh giá, nhận xét theo hai khía cạnh dưới đây. Các số liệu, bảng biểu hay biểu đồ cần phải ghi rõ nguồn số liệu.}
\subsection{Những thành tựu nổi bật}
\subsubsection{* Về chính sách của nhà nước đối với công nghệ  AI}
\subsubsection{* Về sự sẵn sàng của các công ty công nghệ}
\subsubsection{* Về thực tế ứng dụng công ngệ AI vào nền kinh tế và xã hội} 
\textcolor{red}{Nhóm có thể bổ sung thêm vì đây chỉ là những gợi ý.}

\subsection{Những hạn chế, tồn tại }
\subsubsection{* Về hạ tầng công nghệ cho sự phát triển công nghệ AI}
\subsubsection{* Về nguồn nhân lực phát triển công nghệ AI}
\subsubsection{* Về đầu tư cho phát triển công nghệ AI}
\textcolor{red}{Nhóm có thể bổ sung thêm vì đây chỉ là những gợi ý.}
\subsection{Nguyên nhân của những thành tựu, hạn chế }
\subsubsection{* Nguyên nhân của những thành tựu}
Căn cứ vào phần phân tích những thành tựu ở trên để tìm ra những nguyên nhân tương ứng
\subsubsection{* Nguyên nhân của những hạn chế}
Căn cứ vào phần phân tích những hạn chế ở trên để tìm ra những nguyên nhân tương ứng
\section{Những cơ hội và thách thức đối với sự phát triển công nghệ AI ở Việt Nam hiện nay.}
\subsection{Những cơ hội}
\subsubsection{Một là:}
\subsubsection{Hai là:}
\subsubsection{Ba là:}
\textcolor{red}{Nhóm có thể thêm các cơ hội. Mỗi thuận lợi phải phân tích, giải thích tại sao đó là thuận lợi.}

\subsection{Những thách thức }
\subsubsection{Một là:}
\subsubsection{Hai là:}
\subsubsection{Ba là:}
\textcolor{red}{Nhóm có thể thêm các thách thức. Mỗi khó khăn phải phân tích, giải thích tại sao đó là khó khăn.}

\section{Những giải pháp chủ yếu nhằm thúc đẩy sự phát triển của công nghệ AI, đáp ứng yêu cầu của cách mạng công nghiệp lần thứ tư ở Việt Nam trong thời gian tới.}
\subsection{Giải pháp vĩ mô từ phía nhà nước}
\subsubsection{Thứ nhất:}
\subsubsection{Thứ hai:}
\subsubsection{Thứ ba:}
\textcolor{red}{Nhóm có thể thêm các pháp. Mỗi giải pháp đưa ra phải phân tích, giải thích tại sao đó là giải pháp để giải quyết vấn đề.}

\subsection{Giải pháp từ phía các cơ sở đào tạo nguồn nhân lực cho AI}
\subsubsection{Thứ nhất:}
\subsubsection{Thứ hai:}
\subsubsection{Thứ ba:}
\textcolor{red}{Nhóm có thể thêm các pháp. Mỗi giải pháp đưa ra phải phân tích, giải thích tại sao đó là giải pháp để giải quyết vấn đề.}

\subsection{Giải pháp đối từ phía các công ty công nghệ của Việt Nam}
\subsubsection{Thứ nhất:}
\subsubsection{Thứ hai:}
\subsubsection{Thứ ba:}
\textcolor{red}{Nhóm có thể thêm các pháp. Mỗi giải pháp đưa ra phải phân tích, giải thích tại sao đó là giải pháp để giải quyết vấn đề.}

\chapter*{KẾT LUẬN}
\addcontentsline{toc}{chapter}{Kết luận}
\gachdau
Đoạn 1: Nhấn mạnh lại ý nghĩa và vai trò của phát triển công nghệ AI trong quá trình CNH, HĐH ở Việt Nam dưới tác động của cách mạng công nghiệp lần thứ tư\\
\gachdau
Đoạn 2: Khái quát ngắn gọn thực trạng phát triển công nghệ AI của Việt Nam đã phân tích ở trên\\
\gachdau
Đoạn 3: Khái quát những giải pháp thúc đẩy sự phát triển công nghệ AI, đáp ứng yêu cầu của cuộc cách mạng công nghiệp lần thứ tư ở Việt Nam\\
\gachdau
Đoạn 4: Nhấn mạnh những đóng góp và hạn chế của nghiên cứu về sự phát triển của AI ở Việt Nam, gợi ý một số hướng nghiên cứu tiếp theo\\
\gachdau
\textcolor{red}{Chú ý: Phần kết luận không dài quá 1 trang.}

\begin{thebibliography}{99}
\item Bộ Giáo dục và Đào tạo (2021), Giáo trình Kinh tế chính trị Mác - Lênin: Dành cho bậc đại học hệ không chuyên lý luận chính trị, Nhà xuất bản Chính trị Quốc gia - Sự thật, Hà Nội.


\end{thebibliography}
\textcolor{red}{Chú ý: sinh viên phải tìm kiếm thêm những tài liệu tham khảo (dùng công cụ Google Scholar), tối thiểu phải có 10 tài liệu tham khảo. Danh mục tài liệu tham khảo theo tiêu chuẩn APA.}
\end{document}




